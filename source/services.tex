\section{Services}

We can separate services into test and operational services.
Table \ref{tab:testservers} lists known test services supporting
ECH. All of those below .ie and my-own.net were setup by the 
DEfO project.

\small
\begin{longtblr} [
        caption = {Test Services with ECH},
        label = {tab:testservers}
    ] {
        colspec = {| l | p{0.7\linewidth} |},
        rowhead = 1
    }
    \hline
        Name & Details\\
    \hline
        defo.ie & \url{https://defo.ie/ech-check.php} is often used to check ECH\\
        & ECH keys are rotated hourly, private usable for 3 hours, ECHConfigList in DNS contains only latest public\\
    \hline
        draft-13.esni.defo.ie & different server technology instances on different ports as listed at \url{https://defo.ie/}, e.g., \url{https://draft-13.esni.defo.ie:10413} is served by nginx\\
        & ECH keys are rotated hourly, private usable for 3 hours, ECHConfigList in DNS contains only latest public\\
    \hline
        test.defo.ie & hosts a number of ECH server setups with good and variously bad configurations - see
        \url{https://test.defo.ie/iframes_tests} describes those and allows a browser to attempt connections to
        each via Iframes\\
        & ECH keys/configs are static for these setups\\
    \hline
        foo.ie & \url{https://foo.ie/ech-check.php} was setup used to check the defo.ie setup was easily replicated\\
        & ECH keys are rotated hourly, private usable for 3 hours, ECHConfigList in DNS contains only latest public\\
    \hline
        my-own.net & this was test the impact of having the same ECH keys on
        port 443 (\url{https://my-own.net/ech-check.php}) and another port  
        (\url{https://my-own.net:8443/ech-check.php})  - at one point that made a difference to browsers\\
        & ECH keys are rotated hourly, private usable for 3 hours, ECHConfigList in DNS contains only latest public\\
    \hline
        tls-ech.dev & \url{https://tls-ech.dev/} was setup by the boringssl developers as a test server that uses
        boringssl\\ 
        & \todo{Check ECH key rotation}\\
    \hline
        Cloudflare & \url{https://cloudflare-ech.com/cdn-cgi/trace} is a test page
           setup by cloudflare that reports on ECH success/failure\\
        & apparently, the server implementation and infrastucrure are part of Cloudflare's normal setup\\
        & a similar test service used to be available at \url{https://crypto.cloudflare.com/cdn-cgi/trace} but
        that was turned off around the time that ECH was re-enabled for Cloudflare customers\\
        & ECH keys are rotated hourly, private usable for N hours, ECHConfigList in DNS contains only latest public\\
        & \todo{Check what N for private key usability}\\

    \hline
        rfc5746.mywaifu.best & this ECH-enabled web page (\url{https://rfc5746.mywaifu.best/}) seems to have been setup 
        as an ECH test site by someone using DEfO artefacts (nginx and documentation) but without any contact
        having happened between the person who set that up and any of the DEfO-project participants\\
        & the person who set this up documented some of that at \url{https://ckcr4lyf.github.io/tech-notes/services/nginx/nginx-ech.html}\\

    \hline
\end{longtblr}
\normalsize

The only operational ECH services we know of is Cloudflare's deployment.
However, that is non-negligible. Cloudflare briefly enabled ECH in 2022
but soon disabled that again as it caused some back-end issues. They
finally re-enabled ECH in October 2024.

As part of our DEfO test setup, we have a web page where you can enter a DNS
name and port and a script on our server will check if ECH is enabled for a web
server at the name and port. (That's at
\url{https://test.defo.ie/domainechprobe.php}.)
\todo{Add info from that data.}
