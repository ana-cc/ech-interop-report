\section{Interoperability}

Our primary interoperability testing is based on our ``smokeping''-like tests.
While smokeping measures latency for names/addreses, our tests aim to establish
whether and which ECH configurations interoperate or fail to do so.

We mainly do this via a set of hourly cronjobs running on the test.defo.ie VM
that attempt to access 67 URLs (some deliberately with broken configurations)
from six different clients (described in Table \ref{tab:smclients}. We report
results for the most recent 6 runs at
\url{https://test.defo.ie/smokeping-summary.php}.

\small
\begin{longtblr} [
        caption = {Smokeping clients},
        label = {tab:smclients}
    ] {
        colspec = {| l | p{0.7\linewidth} |},
        rowhead = 1
    }
    \hline
        Name & Details\\
    \hline
        chromium & headless browser tests via selenium\\
        & Version: 131.0.6778.85 \\
        & Python script: \url{https://github.com/defo-project/ech-dev-utils/blob/main/scripts/selenium_test.py}\\
        & ECH implementation based on boringssl\\

    \hline
        firefox & headless browser tests via selenium\\
        & Version: 133.0\\
        & Python script: \url{https://github.com/defo-project/ech-dev-utils/blob/main/scripts/selenium_test.py}\\
        & ECH implementation based on NSS\\

    \hline
        curl & bash script using curl\\
        & Version: 8.11.1-DEV\\
        & Bash script: \url{https://github.com/defo-project/ech-dev-utils/blob/main/scripts/smoke_ech_curl.sh}\\
        & curl currently only pays attention the the first HTTPS resource record seen in DNS answers\\
        & ECH implementation based on OpenSSL\\

    \hline
        golang & custom golang programme and bash script\\
        & Version: golang 1.23\\
        & Golang programme: \url{https://github.com/defo-project/ech-dev-utils/blob/main/scripts/ech_url.go}\\
        & Bash script: \url{https://github.com/defo-project/ech-dev-utils/blob/main/scripts/smoke_ech_go.sh}\\
        & ECH implementation developed for golang\\

    \hline
        rustls & custom rustls programme and bash script\\
        & Version: rustls 0.23.19\\
        & Rustls programme: \url{https://github.com/defo-project/ech-dev-utils/blob/main/scripts/ech_url.rs}\\
        & Bash script: \url{https://github.com/defo-project/ech-dev-utils/blob/main/scripts/smoke_ech_rs.sh}\\
        & ECH implementation developed for rustls\\

    \hline
        python & custom Cpython build and python script\\
        & Version: python 3.13\\
        & Python script: \url{https://github.com/defo-project/ech-dev-utils/blob/main/scripts/ech_url.py}\\
        & Bash script: \url{https://github.com/defo-project/ech-dev-utils/blob/main/scripts/smoke_ech_py.sh}\\
        & ECH implementation based on OpenSSL\\

    \hline

\end{longtblr}
\normalsize

There are some notes on these tests worth calling out in case someone wants to
reproduce them:

\begin{itemize}
    \item Some of the headless browser tests (e.g. with badly encoded ECHConfigList values)
        will cause selenium to throw exceptions, so test scripts need to catch those in order
        to properly determine that we got an expected failure
    \item Firefox will not attempt ECH to a name when the hostname of the test machine/VM is
        beneath the 2LD for that name, this causing unexpected failures if that hostname matches
        the origin of a test URL. In our case, we run both servers and test clients on the
        same VM (``test.defo.ie``) so we have to set the hostname on the VM to something
        ``peculiar'' for tests to run as expected. (That of course causes ``sudo'' to 
        complain which could disturb logs.)
    \item In the course of testing ECH in recent years, we have repeatedly made one
        misconfiguration error: using ``.'' as the target name in HTTPS RRs for ports other
        then 443, when the correct thing do is to include the origin as the target name
        in such cases. That is when making an HTTPS RR for \url{https://example.com:12345/}
        one needs to publish an HTTPS RR at ``\_12345.\_https.example.com'' and the 
        (presentation) value of that record needs to start with ``1 example.com ...''
        where 1 is the priority and example.com is that targetName. So simply copying the
        value of the HTTPS RR for port 443 and re-publishing that for another port is
        not sufficient for interoperability. (But also does work for some clients in
        some cases, making it easier to accidentally do this and not notice.)
\end{itemize}

\subsection{Results}

The set of 67 test URLs covers all of the server technologies listed in Table
\ref{tab:servers}) though as all of those use OpenSSL for ECH, we direct most
of the URLs towards one nginx instance, typically both (via a haproxy instance
on) port 443 and direct to the nginx instance on port 15443. The set also
include URLs for the test services listed in Table \ref{tab:testservers}.  Some
test URLs have ``broken'' configurations, e.g. with badly encoded HTTPS
resource records, or have specification-conformant configurations that we
expect may not work.

We have 6 clients testing hourly against 67 URLs, giving us 402 measurements
per hour. The full set of URLs are too long to explain in detail here so
we'll just show current results and describe a few cases where those results
are more ``interesting''. The full set of URLs and the last 6 hours of test
outcomes at \url{https://test.defo.ie/smokeping-summary.php} or 
web pages with (most of) the test URLs loaded via an Iframe can be accessed at
\url{https://test.defo.ie/iframe_tests_sort.html}. That web page also includes
some explanatory text about each test URL.

Table \ref{tab:itests} below shows, for each test URL and each client,
the percentage of expected and total results for a recent two day
interval.~\footnote{Table \ref{tab:itests} is produced
using \url{https://github.com/defo-project/ech-dev-utils/blob/main/scripts/smokeping-summary-report.py}.}

\todo{Check expected values and sort out any self-caused anomalies.}

\input tt

The set of ``interesting'' things arising from Table \ref{tab:itests} include

\begin{itemize}
\item Line 2 - \url{https://2thenp-ng.test.defo.ie:15443/echstat.php?format=json} works for all
    clients except chromium where it never works, yet the same configuration on port 443 works
    100\% of the time on all clients. That configuration has an ECHConfigList with two
    entries - the first an x25519 key and the second a p256 key - and the same ECHConfigList 
    value is published for both ports. The error here however is in the configuration - the
    targetName for the port 15443 RR is ``.'' rather than the hostname, so chromium here
    is correct and our test setup is wrong. We fixed a bug in 
    \url{https://github.com/defo-project/ech-dev-utils/blob/main/test-cases/test_cases_gen.py}
        addressing this on 20241208.\\
    \todo{publish a ``bad'' config like that as a specific test}
\item Lines 5 and 6 - \url{https://badalpn-ng.test.defo.ie/echstat.php?format=json} and
    it's equivalent off port 443, work fine with all clients except Firefox and never work
    for firefox. The relevant DNS RR has an illegal alpn value (a '+' character). Hard to
    see this as other than firefox being over-fussy, though there could of course be some
    good reason to reject such alpn values.
\item Line 7 and elsewhere - fixed a bug in 
    \url{https://github.com/defo-project/ech-dev-utils/blob/main/scripts/selenium_test.py} where
        expected failures from tests using the ``echstat\_stat()'' handler were being logged as 
        test failures when they should be logged as expected.
    \item Line 13 - our ``cf\_check'' result handler hadn't been changed to be applied 
        with the change in Cloudflare's ECH test URL from ``crypto.cloudflare.com'' to
        ``cloudflare-ech.com'' which caused erroneous failure reports.
    \item Line 13 - our python test client is throwing an exception for the Cloudflare
        test URL:
        \begin{verbatim}
[SSL: SSLV3_ALERT_ILLEGAL_PARAMETER] ssl/tls alert illegal parameter (_ssl.c:1024)
        \end{verbatim}
        being investigated...
    \item Line 20 and similar - our python test client needed a tweak to not fail
        for tests not using the ``echstat.php'' server side script - fix was pushed
        on 20241209
    \item Line 32 - somehow certbot was no longer installed on test.defo.ie so we
        got a cert expiry for hidden.hoba.ie - re-instaled certbot to fix, which
        leads to some redundancy as we end up with >1 cert for some names (more or
        less all other name on that VM are below *.test.defo.ie)
    \item Line 35 - out python test client hits a certificate validation error for
        this one:
        \begin{verbatim}
[SSL: CERTIFICATE_VERIFY_FAILED] certificate verify failed: unable to get local issuer certificate (_ssl.c:1024)
        \end{verbatim}
        being investigated...
        
\end{itemize}

\todo{based on the above, re-generate Table \ref{tab:itests} in some hours}
