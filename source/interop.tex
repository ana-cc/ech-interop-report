\section{Interoperability}

Our primary interoperability testing is based on our ``smokeping''-like tests.
While smokeping measures latency for names/addreses, our tests aim to establish
whether and which ECH configurations interoperate or fail to do so.

We mainly do this via a set of hourly cronjobs running on the test.defo.ie VM
that attempt to access 67 URLs (some deliberately with broken configurations)
from six different clients (described in Table \ref{tab:smclients}. We report
results for the most recent 6 runs at
\url{https://test.defo.ie/smokeping-summary.php}.

\small
\begin{longtblr} [
        caption = {Smokeping clients},
        label = {tab:smclients}
    ] {
        colspec = {| l | p{0.7\linewidth} |},
        rowhead = 1
    }
    \hline
        Name & Details\\
    \hline
        chromium & headless browser tests via selenium\\
        & Version: 131.0.6778.85 \\
        & Python script: \url{https://github.com/defo-project/ech-dev-utils/blob/main/scripts/selenium_test.py}\\
        & ECH implementation based on boringssl\\

    \hline
        firefox & headless browser tests via selenium\\
        & Version: 133.0\\
        & Python script: \url{https://github.com/defo-project/ech-dev-utils/blob/main/scripts/selenium_test.py}\\
        & ECH implementation based on NSS\\

    \hline
        curl & bash script using curl\\
        & Version: 8.11.1-DEV\\
        & Bash script: \url{https://github.com/defo-project/ech-dev-utils/blob/main/scripts/smoke_ech_curl.sh}\\
        & curl currently only pays attention the the first HTTPS resource record seen in DNS answers\\
        & ECH implementation based on OpenSSL\\

    \hline
        golang & custom golang programme and bash script\\
        & Version: golang 1.23\\
        & Golang programme: \url{https://github.com/defo-project/ech-dev-utils/blob/main/scripts/ech_url.go}\\
        & Bash script: \url{https://github.com/defo-project/ech-dev-utils/blob/main/scripts/smoke_ech_go.sh}\\
        & ECH implementation developed for golang\\

    \hline
        rustls & custom rustls programme and bash script\\
        & Version: rustls 0.23.19\\
        & Rustls programme: \url{https://github.com/defo-project/ech-dev-utils/blob/main/scripts/ech_url.rs}\\
        & Bash script: \url{https://github.com/defo-project/ech-dev-utils/blob/main/scripts/smoke_ech_rs.sh}\\
        & ECH implementation developed for rustls\\

    \hline
        python & custom Cpython build and python script\\
        & Version: python 3.13\\
        & Python script: \url{https://github.com/defo-project/ech-dev-utils/blob/main/scripts/ech_url.py}\\
        & Bash script: \url{https://github.com/defo-project/ech-dev-utils/blob/main/scripts/smoke_ech_py.sh}\\
        & ECH implementation based on OpenSSL\\

    \hline

\end{longtblr}
\normalsize

There are some notes on these tests worth calling out in case someone wants to
reproduce them:

\begin{itemize}
    \item Some of the headless browser tests (e.g. with badly encoded ECHConfigList values)
        will cause selenium to throw exceptions, so test scripts need to catch those in order
        to properly determine that we got an expected failure
    \item Firefox will not attempt ECH to a name when the hostname of the test machine/VM is
        beneath the 2LD for that name, this causing unexpected failures if that hostname matches
        the origin of a test URL. In our case, we run both servers and test clients on the
        same VM (``test.defo.ie``) so we have to set the hostname on the VM to something
        ``peculiar'' for tests to run as expected. (That of course causes ``sudo'' to 
        complain which could disturb logs.)
    \item In the course of testing ECH in recent years, we have repeatedly made one
        misconfiguration error: using ``.'' as the target name in HTTPS RRs for ports other
        then 443, when the correct thing do is to include the origin as the target name
        in such cases. That is when making an HTTPS RR for \url{https://example.com:12345/}
        one needs to publish an HTTPS RR at ``\_12345.\_https.example.com'' and the 
        (presentation) value of that record needs to start with ``1 example.com ...''
        where 1 is the priority and example.com is that targetName. So simply copying the
        value of the HTTPS RR for port 443 and re-publishing that for another port is
        not sufficient for interoperability. (But also does work for some clients in
        some cases, making it easier to accidentally do this and not notice.)
\end{itemize}

\subsection{Results}

The set of 67 test URLs covers all of the server technologies listed in Table
\ref{tab:servers}) though as all of those use OpenSSL for ECH, we direct most
of the URLs towards one nginx instance, typically both (via a haproxy instance
on) port 443 and direct to the nginx instance on port 15443. The set also
include URLs for the test services listed in Table \ref{tab:testservers}.  Some
test URLs have ``broken'' configurations, e.g. with badly encoded HTTPS
resource records, or have specification-conformant configurations that we
expect may not work.

We have 6 clients testing hourly against 67 URLs, giving us 402 measurements
per hour. The full set of URLs are too long to explain in detail here so
we'll just show current results and describe a few cases where those results
are more ``interesting''. The full set of URLs and the last 6 hours of test
outcomes at \url{https://test.defo.ie/smokeping-summary.php} or 
web pages with (most of) the test URLs loaded via an Iframe can be accessed at
\url{https://test.defo.ie/iframe_tests_sort.html}. That web page also includes
some explanatory text about each test URL.

Table \ref{tab:itests} below shows, for each test URL and each client,
the percentage of expected and total results for a recent two day
interval.~\footnote{Table \ref{tab:itests} is produced
using \url{https://github.com/defo-project/ech-dev-utils/blob/main/scripts/smokeping-summary-report.py}.}


\input tt

The set of ``interesting'' things arising from Table \ref{tab:itests} include:

\todo{Check text still applies after we've updated Table \ref{tab:itests}}

\begin{itemize}

    \item Lines 5 and 6 -
        \url{https://badalpn-ng.test.defo.ie/echstat.php?format=json} and it's
        equivalent off port 443, work fine with all clients except Firefox and
        never work for firefox. The relevant DNS RR has an illegal alpn value
        (a '+' character). Hard to see this as other than firefox being
        over-fussy, though there could of course be some good reason to reject
        such alpn values.

    \item Lines 13 and 57 - our python test client throws an exception for
        these (the Cloudflare operational, and boringssl test, servers)
        \begin{verbatim} [SSL: SSLV3_ALERT_ILLEGAL_PARAMETER] ssl/tls alert
        illegal parameter (_ssl.c:1024) \end{verbatim} This is being
        investigated.

    \item Line 14 -
        \url{https://curves1-ng.test.defo.ie/echstat.php?format=json} has two
        HTTPS RRs published with the same priority, one that contains an x25519
        public value and another that only have a p256 public value. This seems
        to cause failure for our rustls client about 50\% of the time as that
        client only supports x25519 (presumably). Browsers, while similarly not
        supporting NIST curves, presumably either select the ``working'' HTTPS
        RR value, and/or use the retry-configs fallback and manage to connect
        with ECH.

    \item Line 15 becomes a puzzle given line 14 - seemingly our rustls client
        does always work when not on port 443! We have verified that the HTTPS
        RRs published for curves1-ng.test.defo.ie and
        \_15443.\_https.curves1-ng.test.defo.ie are as expected. Being
        investigated.

    \item Lines 18 and 19 may behave similarly to the above, being alike except
        that the HTTPS RR value priority for the p256 case is lower than for
        the 25519 one.

    \item Line 23 - browsers seem to dislike
        \url{https://draft-13.esni.defo.ie:12413/}.  We suspect that's due to
        our test result handlers. That test case is an ECH shared-mode hapoxy
        instance with lighttpd as the backend. As lighttpd is the backend it
        doesn't produce an HTTP response indicates ECH worked as ECH was
        terminated with TLS at the haproxy intance. It's unclear how we can
        handle this test case when browsers don't indicate ECH success or
        failure. (Between haproxy and lighttpd we may be able to engineer some
        HTTP response header that indicates ECH success or failure - that'd not
        be secure but would be fine for testing. That's not been attempted
        yet.)

    \item Line 37 - it's not clear why publishing many (20) identical HTTPS RRs
        would work for port 443 for browsers but not off port 443. To be
        investigated.

\end{itemize}

\todo{based on the above, re-generate Table \ref{tab:itests} in some hours}

\subsection{Recently Fixed Issues with Test Setup}

It may be illustrative to describe some self-caused test failures seen as we
were preparing this report, and how those were resolved:

\begin{itemize}

\item Line 2 -
    \url{https://2thenp-ng.test.defo.ie:15443/echstat.php?format=json} works
        for all clients except chromium where it never works, yet the same
        configuration on port 443 works 100\% of the time on all clients. That
        configuration has an ECHConfigList with two entries - the first an
        x25519 key and the second a p256 key - and the same ECHConfigList value
        is published for both ports. The error here however is in the
        configuration - the targetName for the port 15443 RR is ``.'' rather
        than the hostname, so chromium here is correct and our test setup is
        wrong. We fixed a bug in
        \url{https://github.com/defo-project/ech-dev-utils/blob/main/test-cases/test_cases_gen.py}
        addressing this on 20241208.\\

    \item Lines 5 and 6 -
        \url{https://badalpn-ng.test.defo.ie/echstat.php?format=json} and it's
        equivalent off port 443, work fine with all clients except Firefox and
        never work for firefox. The relevant DNS RR has an illegal alpn value
        (a '+' character). Hard to see this as other than firefox being
        over-fussy, though there could of course be some good reason to reject
        such alpn values.

    \item Line 7 and elsewhere - fixed a bug in
        \url{https://github.com/defo-project/ech-dev-utils/blob/main/scripts/selenium_test.py}
        where expected failures from tests using the ``echstat\_stat()''
        handler were being logged as test failures when they should be logged
        as expected.

    \item Line 13 - our ``cf\_check'' result handler hadn't been changed to be
        applied with the change in Cloudflare's ECH test URL from
        ``crypto.cloudflare.com'' to ``cloudflare-ech.com'' which caused
        erroneous failure reports.

    \item Line 20 and similar - our python test client needed a tweak to not
        fail for tests not using the ``echstat.php'' server side script - fix
        was pushed on 20241209

    \item Line 32 - somehow certbot was no longer installed on test.defo.ie so
        we got a cert expiry for hidden.hoba.ie - re-instaled certbot to fix,
        which leads to some redundancy as we end up with more than one cert for
        some names (more or less all other name on that VM are below
        *.test.defo.ie)

    \item Line 35 - our python test client hits a certificate validation error
        for this lighttpd server - a change to the server configuration seems
        to fix this, the change being to send the full cert chain to clients
        (browsers don't need that but our other test client do).

    \item Lines 51 and 52: these needed an edit for the expected values, which were
        wrong.

\end{itemize}

