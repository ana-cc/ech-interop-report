\section{Introduction}

Deployments of the Transport Layer Security (TLS~\cite{rfc8446}) protocol
expose the name of the server (e.g. the web site DNS name) via the Server Name
Indication (SNI) field in the first message sent (the ClientHello).  The
Encrypted Client Hello (ECH)~\cite{draft-ietf-tls-esni} extension to TLS is a
privacy-enhancing scheme that aims to address this leak.
This report describes the current state of ECH interoperability.
The primary audience for this document are those implementing and
deploying ECH. Secondarily, there may be lessons to learn for those
designing protocols like ECH.

The Open Technology Fund (OTF, \url{https://www.opentech.fund/}) have
funded the DEfO project (\url{https://defo.ie}) to develop
ECH implementations for OpenSSL, and to otherwise encourage implementation
and deployment of ECH.
As we expect the implementation and deployment environment for ECH to change
over time, this report will be updated as events warrant and is currently
versioned based on the build-time of this PDF.

The first version of this report was published in January 2025. \cite{echinterop1}
For this version we updated all software to latest versions and re-ran tests.
\todo{finish updates, not all done yet}

The latest version can be found at
\url{https://github.com/defo-project/ech-interop-report/blob/main/ech-interop-report.pdf}.
Comments, additions or corrections are welcome. Those can be sent via email to
the author, or as issues or PRs (preferably for the latex files in the
``source2'' directory) in the github repository for this report which is
\url{https://github.com/defo-project/ech-interop-report}.
