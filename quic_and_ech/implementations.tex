\section{Existing Implementations}

Several open-source QUIC implementations exist. This report considers the Open Source implementations that are actively evaluated by the QUIC interoperability runner~\cite{interop}, which performs automated tests to ensure the implementations are compatible with each other. Differences may arise where the standard has not given enough detail on a specific aspect of the protocol design, and this testing is expected to highlight differences in interpretation.

ECH support was determined from analysing the source code for these implementations in Table~\ref{tbl:quics-ech}.

\begin{table}[ht]
\centering
\begin{tabular}{l|l|l|l}
QUIC Impl. &  TLS Library & Developer & ECH Support  \\
\hline \hline
\color{red}{\texttt{mvfst}} &  fizz & Facebook & \color{red}{Yes} \\
\hline
\color{red}{\texttt{neqo}}  & NSS & Mozilla & \color{red}{Yes} \\
\hline
\color{red}{\texttt{chrome}} & BoringSSL & Google & \color{red}{Yes} \\
\hline
\texttt{quiche}   & BoringSSL & Cloudflare & No \\
\hline
\texttt{msquic}   & OpenSSL & Microsoft & No \\
\hline
\texttt{xquic}   & BoringSSL, BabaSSL& Alibaba & No \\
\hline
\texttt{s2n-quic}   & rustls, s2n-tls & AWS & No \\
\hline
\texttt{lsquic}   & BoringSSL & LiteSpeed & No \\
\hline
\color{red}{\texttt{ngtcp2}}    & BoringSSL & T. Tsujikawa & \color{red}{Yes}  \\
\hline
\texttt{picoquic} & OpenSSL & C. Huitema & No  \\
\hline
\texttt{quic-go}  & Go Std. Lib.& M. Seemann& No  \\
\hline
\texttt{kwik} & Agent15 & P. Doornbosch& No \\
\hline
\texttt{aioquic}   & OpenSSL & J. Lainé & No  \\
\hline
\texttt{quinn}   & rustls & D. Ochtman, B.Saunders & No  \\

\end{tabular}
\caption{ECH Support across QUIC implementations}
\label{tbl:quics-ech}
\end{table}

Before starting the handshake, QUIC also has to provide TLS with the transport parameters~\footnote{\url{https://www.iana.org/assignments/quic/quic.xhtml}} that it wishes to carry. This is an additional requirement from TCP~\cite{rfc9000}. As per RFC 9001, the \texttt{quic\_transport\_parameters} extension has to be carried in a TLS ClientHello and the EncryptedExtensions messages during the handshake. 
Differences regarding transport parameter handling exist within the four QUIC implementations that were found to support ECH, with evidence that \texttt{neqo} allows different transport parameters in the inner and outer ClientHello, however cautioning against using this feature unless the ECH configuration has been validated, due to a mechanism that filters sensitive transport parameters in the outer ClientHello (see \href{https://github.com/mozilla/neqo/blob/main/neqo-crypto/src/agent.rs#L1176-L1182}{source code}).
For the other implementations (\texttt{mvfst}, \texttt{chrome}, and \texttt{ngtcp2}), this behaviour appears to depend on the underlying TLS library, however no mechanism for filtering sensitive transport parameters was noted in the libraries BoringSSL and fizz.

%I think BoringSSL allows the transport parameter extension in the Outer ClientHello (see https://github.com/google/boringssl/blob/main/ssl/extensions.cc#L3640-L3646 and https://github.com/google/boringssl/blob/main/ssl/encrypted_client_hello.cc#L875-L880), but fizz does not.
